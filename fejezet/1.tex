\section{Igénybevételek típusai (befogott rúdra)}%###############################################################

\begin{outline}
	\1 $R_\text{eh}$ -- folyáshatár
	\1 $\sigma_\text{N} = \frac{F}{A}\le\sigma\text{meg}=\frac{R_\text{eh}}{n}$
	\1 húzó-nyomó
		\2 $\Delta L = \frac{FL}{AE}$
		\2 $\epsilon = \frac{\Delta L}{L}$
		\2 $\epsilon_\text{kereszt} = \nu\epsilon_\text{hossz}$
		\2 Hooke törvény: $\sigma=E\epsilon$
	\1 nyírás
		\2 $\tau_\text{v} = \frac{4}{3}\frac{SF}{I_\text{x}s} \approx \frac{F}{A}$
	\1 hajlítás
		\2 $\sigma_\text{H} = \frac{M_\text{H}}{I}e$
		\2 lehajlás: $f = \frac{FL^3}{3IE}$
		\2 téglalap alapú hasábra $I_\text{xy}=\frac{ab^3}{12}$
		\2 kör alapú hasábra $I_\text{xy}=\frac{d^4\pi}{64}$
	\1 csavarás
		\2 $\tau_\text{t} = \frac{M_\text{t}}{I_\text{p}}e$
		\2 elcsavarodás: $\varphi = \frac{M_\text{t}L}{I_\text{p}G}$
		\2 kör alapú hasábra $I_\text{p}=I_\text{x}+I_\text{y}=\frac{d^4\pi}{32}$
	\1 felületi nyomás
		\2 $p = \frac{F}{A_\perp}\le p_\text{meg}$
	\1 összetett igénybevétel
		\2 HMH: $\sigma_\text{egy} = \sqrt{\sigma^2+3\tau^2}\le\sigma_\text{meg}$
		\2 Mohr $\sigma_\text{egy} = \sqrt{\sigma^2+4\tau^2}\le\sigma_\text{meg}$
		\2 húzás vagy hajlítás esetén $\sigma = \sigma_\text{N}+\sigma_\text{H}$
		\2 nyírás vagy csavarás esetén csak a domináns igénybevétellel számolunk
\end{outline}
